\documentclass[finnish]{../tktltiki2}

% --- General packages ---

\usepackage[utf8]{inputenc}
\usepackage{lmodern}
\usepackage{microtype}
\usepackage{amsfonts,amsmath,amssymb,amsthm,booktabs,color,enumitem,graphicx}
\usepackage[pdftex,hidelinks]{hyperref}

% Automatically set the PDF metadata fields
\makeatletter
\AtBeginDocument{\hypersetup{pdftitle = {\@title}, pdfauthor = {\@author}}}
\makeatother

% --- Language-related settings ---

\usepackage[fixlanguage]{babelbib}
\selectbiblanguage{finnish}

\usepackage[nottoc,numbib]{tocbibind}
\settocbibname{Lähteet}

% --- Theorem environment definitions ---

\newtheorem{lau}{Lause}
\newtheorem{lem}[lau]{Lemma}
\newtheorem{kor}[lau]{Korollaari}

\theoremstyle{definition}
\newtheorem{maar}[lau]{Määritelmä}
\newtheorem{ong}{Ongelma}
\newtheorem{alg}[lau]{Algoritmi}
\newtheorem{esim}[lau]{Esimerkki}

\theoremstyle{remark}
\newtheorem*{huom}{Huomautus}


% --- tktltiki2 options ---

\title{Opiskelijoiden ohjelmointitaidon arviointi \\ohjelmoinnin peruskurssilla}
\author{Tomi Simsiö}
\date{\today}
\level{Aine}

\begin{document}

% --- Front matter ---

\maketitle
\tableofcontents
\newpage


% --- Main matter ---

\section{Johdanto}
Ohjelmointitaidon mittaaminen on vaikeaa. Yleisesti käytössä olevat paperikokeet ovat osoittautuneet riittämättömäksi ohjelmointitaidon mittariksi~\cite{MAD01}. Opiskelijat eivät osaa tuottaa toimivia ohjelmia vaikka ovat läpäisseet alkeiskurssin. Siksi paperikokeiden rinnalle on kokeiltu ottaa käyttöön muita arviointitapoja~\cite{CJ01, CG02, WM03}.

\section{Paperikokeet}
Paperikokeet ovat perinteinen tapa arvioida opiskelijan tietoja. Petersen, Craig ja Zingaro tutkivat 15 ohjelmoinnin peruskurssin (CS1) koetta 14 eri yliopistosta~\cite{PCZ11}.

...

\section{Ohjelmointikokeet}
Ohjelmointikokeissa (lab practicum, lab exam) opiskelijat toteuttavat ja testaavat rajatussa ajassa tehtävänannon mukaisen ohjelman ilman ulkopuolista apua. Käytössään heillä on lehtinen kurssin aikana opetetuista asioista. Kokeen valvojat eivät anna apua ongelman ratkaisua koskeviin kysymyksiin. Ohjelmointikokeella pyritään saavuttamaan paperikokeen korvaava tai sitä täydentävä arviointi, joka ei mittaa muuta kuin opiskelijan ohjelmointitaitoa.

Chamillard ja Braun esittelevät tämänlaista arviointia artikkelissaan "Evaluating Programming Ability in an Introductory Computer Science Course"~\cite{CB00}. He pitivät ohjelmoinnin peruskurssin aikana kaksi ohjelmointikoetta: ensimmäinen kurssin puolivälissä ja toinen noin kolme viikkoa ennen kurssin loppua.

Chamillard ja Braun kertovat näiden kokeiden aiheuttavan stressiä opiskelijoissa. Tavallisen kokeeseen liittyvän stressin lisäksi painetta aiheuttaa tarve saada ohjelma valmiiksi aikarajan sisällä. Vaikka ohjelmointikokeet ovat samanlaisia yhdessä tehtävien harjoitustöiden kanssa, joita opiskelijat ovat jo suorittaneet, ovat Chamillard ja Braun havainneet opiskelijoiden saavan ohjelmointikokeista huomattavasti heikompia arvosanoja. He myöskin huomasivat opiskelijoiden kaipaavan enemmän apua kurssin asioissa, joita he eivät ole ymmärtäneet henkilökohtaisten harjoitustöiden lähestyessä. Henkilökohtaiset harjoitustyöt näyttävät siis toimivan hyvän ohjelmointitaidon arviointikeinon lisäksi opiskelijoille kannustimena ymmärtää ohjelmointikäsitteitä.

Myös Daly ja Waldron vertailevat artikkelissaan "Assessing the assessment of programming ability" ohjelmointikoetta muihin arviointikeinoihin~\cite{DW04}. Heidän kokeissaan opiskelijoiden tulee toteuttaa lyhyet ratkaisut kolmeen erilliseen tehtävään. ...


\section{Ryhmätyöt}
Chamillardin ja Braunin järjestämällä kurssilla ryhmätyö suoritetaan lukukauden loppupuolella 2-4 hengen ryhmissä. Siinä tutustutetaan opiskelijat ryhmädynamiikkaan ja tarjoaa kokemuksen projektista, joka on liian iso yhden opiskelijan tehtäväksi.

Chamillard ja Braun tunnustavat, että ryhmätyö ei sisällä uutta opetussisältöä ja täten on epäselvää kehittääkö se opiskelijoiden ohjelmointitaitoa. Palautteen mukaan ryhmätyö motivoi opiskelijoita, joten Chamillard ja Braun uskovat sen tarjoaman kokemuksen olevan hyödyllinen. Yhdessä tehtävän harjoitustyön tavoin ryhmätyökään ei mahdollista yksilöiden ohjelmointitaidon tarkkaa arvioimista.

\section{Muut arviointimenetelmät}


\section{Arviointimenetelmien arviointi}


\section{Yhteenveto}



% --- Back matter ---

\bibliographystyle{babplain}
\bibliography{../lahteet}


\end{document}