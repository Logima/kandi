\documentclass[finnish]{beamer}
\usepackage{lmodern}
\usepackage[utf8]{inputenc}
\usepackage{url}
\usepackage{babel}

\author{Tomi Simsiö}
\title{Opiskelijoiden ohjelmointitaidon arviointi \texorpdfstring{\\}{}ohjelmoinnin peruskurssilla}
\institute{HELSINGIN YLIOPISTO\\Tietojenkäsittelytieteen laitos}
\date{27. maaliskuuta 2013}

\begin{document}

\frame{\titlepage}

\frame
{
  \frametitle{Ohjelmointitaidon arviointimenetelmät}
  \begin{itemize}
    \item Paperikokeet
    \item Ohjelmointikokeet
    \item Ryhmätyöt
    \item Harjoitustyöt
  \end{itemize}
}

\frame
{
  \frametitle{Paperikokeet}
  \begin{itemize}
    \item Perinteinen arviointitapa ohjelmointikursseilla.
    \item Osoitettu riittämättömäksi ohjelmointitaidon mittariksi.
  \end{itemize}
}

\frame
{
  \frametitle{Paperikokeet}
  \begin{table}
    \begin{tabular}{ l c r }
      program structure & function structure & data flow \\
      trivial syntax & variables & expressions \\
      data types & booleans & if-else \\
      sequential control flow & input/output & definite loops \\
      indefinite loops & nested loops & strings \\
      lists (arrays) & collections & trees \\
      recursion & exceptions & object structure \\
      basic OO & accessors/mutators & reference types \\
      constructors & inheritance & polymorphism \\
      visibility/scope \\
    \end{tabular}
  \caption{Alkeiskurssin 28 oleellisinta käsitettä}
  \label{tab:kasitteet}
  \end{table}
}

\frame
{
  \frametitle{Ohjelmointikokeet}
  \begin{itemize}
    \item Tehtävänä toteuttaa ja testata rajatussa ajassa tehtävänannon mukainen ohjelma ilman ulkopuolista apua.
    \item Pyritään saavuttamaan paperikokeen korvaava tai sitä täydentävä arviointi.
    \item Mittaa ainoastaan opiskelijan ohjelmointitaitoa.
  \end{itemize}
}

\frame
{
  \frametitle{Ryhmätyöt}
  \begin{itemize}
    \item Ei sovellu yksilöiden ohjelmointitaidon arviointiin.
    \item Antaa kokemuksen isommasta projektista.
  \end{itemize}
}

\frame
{
  \frametitle{Harjoitustyöt}
  \begin{itemize}
    \item Vaikeusaste ja tekemiseen tarvittava aika vaihtelee.
    \item Keskustelu ja yhteistyö sallittua.
    \item Jokaisen täytyy palauttaa oma henkilökohtainen työnsä.
  \end{itemize}
}

\frame
{
  \frametitle{Arviointimenetelmien arviointi}
  \begin{table}
    \begin{tabular}{ l | c c }
       & Keskiarvo & Keskihajonta \\
      Harjoitustyöt & 90,81 & 23,40 \\
      Ryhmätyö & 91,78 & 8,55 \\
      Ohjelmointikokeet & 77,90 & 14,85 \\
      Testit & 73,56 & 11,31 \\
      Loppukoe & 74,99 & 16,84 \\
    \end{tabular}
  \caption{Arviointimenetelmien keskimääräiset arvosanat}
  \label{tab:arviointimenetelmat}
  \end{table}
}

\frame
{
  \frametitle{Arviointimenetelmien arviointi}
  \begin{table}
    \begin{tabular}{ l c }
      Pari & Korrelaatio \\
      \hline
      Harjoitustyöt/Ryhmätyö & 0,249 \\
      Harjoitustyöt/Ohjelmointikokeet & 0,569 \\
      Harjoitustyöt/Testit & 0,603 \\
      Harjoitustyöt/Loppukoe & 0,484 \\
      Ryhmätyö/Ohjelmointikokeet & 0,155 \\
      Ryhmätyö/Testit & 0,193 \\
      Ryhmätyö/Loppukoe & 0,145 \\
      Ohjelmointikokeet/Testit & 0,618 \\
      Ohjelmointikokeet/Loppukoe & 0,593 \\
      Testit/Loppukoe & 0,679 \\
    \end{tabular}
  \caption{Pearsonin korrelaatiokertoimet arviointimenetelmien välillä}
  \label{tab:korrelaatiot}
  \end{table}
}

\frame
{
  \frametitle{Kysymyksiä?}
}

\end{document}
