\documentclass[finnish]{../tktltiki2}

% --- General packages ---

\usepackage[utf8]{inputenc}
\usepackage{lmodern}
\usepackage{microtype}
\usepackage{amsfonts,amsmath,amssymb,amsthm,booktabs,color,enumitem,graphicx}
\usepackage[pdftex,hidelinks]{hyperref}

% Automatically set the PDF metadata fields
\makeatletter
\AtBeginDocument{\hypersetup{pdftitle = {\@title}, pdfauthor = {\@author}}}
\makeatother

% --- Language-related settings ---

\usepackage[fixlanguage]{babelbib}
\selectbiblanguage{finnish}

\usepackage[nottoc,numbib]{tocbibind}
\settocbibname{Lähteet}

% --- Theorem environment definitions ---

\newtheorem{lau}{Lause}
\newtheorem{lem}[lau]{Lemma}
\newtheorem{kor}[lau]{Korollaari}

\theoremstyle{definition}
\newtheorem{maar}[lau]{Määritelmä}
\newtheorem{ong}{Ongelma}
\newtheorem{alg}[lau]{Algoritmi}
\newtheorem{esim}[lau]{Esimerkki}

\theoremstyle{remark}
\newtheorem*{huom}{Huomautus}


% --- tktltiki2 options ---

\title{Opiskelijoiden ohjelmointitaidon arviointi\\ohjelmoinnin peruskurssilla}
\author{Tomi Simsiö}
\date{\today}
\level{Referaatti}

\begin{document}

% --- Front matter ---

\maketitle
\tableofcontents
\newpage


% --- Main matter ---

\section{Johdanto}

Ohjelmointitaidon mittaaminen on vaikeaa. Syynä tähän ei ole keinojen puuttuminen vaan se, että ohjelmointitaitoa pystytään mittaamaan niin monella tavalla, että parhaan tavan valitseminen kuhunkin tilanteeseen on vaikeaa. Chamillard ja Braun esittelevät U.S. Air Force Academyn ohjelmoinnin peruskurssilla käytettyjä ohjelmointitaidon arviointitekniikoita artikkelissaan "Evaluating Programming Ability in an Introductory Computer Science Course"~\cite{CB00}. Tässä referaatissa käydään läpi Chamillardin ja Braunin esittämiä arviointitekniikoita.

\section{Arviointitekniikat}

Kurssin 1000 pistettä jakautuvat eri tehtävien välillä seuraavasti: yhdessä tehtävät harjoitustyöt (collaborative labs) (180 pistettä), ryhmäprojekti (group case study) (75 pistettä), henkilökohtaiset harjoitustyöt (lab practica) (160 pistettä), testit (300 pistettä), loppukoe (250 pistettä) ja muu aktiivisuus (35 pistettä). Viimeisintä lukuunottamatta kaikki näistä arvioivat opiskelijan ohjelmointitaitoa jollain tapaa.

\begin{description}

\item[Yhdessä tehtävät harjoitustyöt] \hfill

Ensimmäinen ohjelmointitaidon arviointimetodi on yhdessä tehtävät harjoitustyöt. Opiskelijat tekevät kurssin aikana 6 harjoitustyötä, joista neljälle on varattu 2 tuntia ohjattua aikaa luokassa. Harjoitusten vaikeusaste vaihtelee helpoimmasta, jossa asetetaan arvoja muuttujiin ja hallitaan yksinkertaisia lukuja ja tulostuksia, vaikeimpaan harjoitukseen, jossa käytetään taulukoita sekä tiedostolukua ja -kirjoitusta. Harjoitustöiden aikana opiskelijat saavat puhua ja tehdä yhteistyötä toistensa kanssa, mutta jokaisen täytyy palauttaa oma henkilökohtainen työnsä arvostelua varten.

Chamillardin ja Braunin mukaan yhdessä tehtävät harjoitustyöt tekevät ohjelmoinnin oppimisesta mahdollisimman helppoa. Vaikka paikalla olisi ohjaajiakin, on hyödyllistä sallia opiskelijoiden kysyä neuvoa myös toisiltaan. Joidenkin opiskelijoiden on havaittu oppivan paremmin muiden kanssa. Yhdessä tehtävät harjoitustyöt tarjoavat näille opiskelijoille tehokkaan oppimisympäristön. Muuten hyvän metodin varjopuolena on vaikea arviointi, joka ei anna tarkkaa kuvaa yksilön taitotasosta.

\item[Ryhmätyö] \hfill

Ryhmätyö suoritetaan lukukauden loppupuolella 2-4 hengen ryhmissä. Ryhmätyö tutustuttaa opiskelijat ryhmädynamiikkaan ja tarjoaa kokemuksen projektista, joka on liian iso yhden opiskelijan tehtäväksi. Ryhmät toteuttavat yleensä jonkin pelin. Kurssin viimeisenä päivänä järjestetään turnaus, jossa opiskelijat pelaavat toisiaan vastaan omia pelejään ja kurssipisteitä jaetaan sijoitusten mukaan. Tämä on innostanut vähemmän lahjakkaitakin opiskelijoita kehittämään strategioita turnauksessa pärjäämiseen.

Chamillard ja Braun tunnustavat että ryhmätyö ei sisällä uutta opetussisältöä ja on siksi epäselvää kehittääkö se opiskelijoiden ohjelmointitaitoa. Palautteen mukaan ryhmätyö motivoi opiskelijoita, joten Chamillard ja Braun uskovat sen tarjoaman kokemuksen olevan hyödyllinen. Yhdessä tehtävän harjoitustyön tavoin ryhmätyökään ei mahdollista yksilöiden ohjelmointitaidon tarkkaa arvioimista.

\item[Henkilökohtaiset harjoitustyöt] \hfill

Vaikka yhdessä tehtävissä harjoitustöissä ja ryhmätyössä on ryhmä\-työskentelyn tuomat edut, yksittäisen opiskelijan arviointi on vaikeaa, Chamillard ja Braun selostavat. Tämän ongelman ratkaisuna he käyttävät kurssilla kahta henkilökohtaista harjoitustyötä.

Henkilökohtaisissa harjoitustöissä opiskelijoiden tulee tehdä ja testata annetun ongelman ratkaiseva ohjelma 85 minuutin aikana. Avukseen he saavat lehtisen, jossa on listattu kurssilla läpikäydyt ohjelmointirakenteet (programming constructs), sekä yleisimmät ohjelmointivirheet ratkaisuineen. Muun materiaalin käyttö on kiellettyä, eivätkä ohjaajat anna apua ongelman ratkaisua koskeviin kysymyksiin. Ensimmäinen henkilökohtainen harjoitustyö pidetään kurssin puolivälissä ja toinen noin kolme viikkoa ennen kurssin loppua.

Chamillard ja Braun kertovat näiden harjoitustöiden aiheuttavan stressiä opiskelijoissa. Tavallisen kokeeseen liittyvän stressin lisäksi painetta aiheuttaa tarve saada ohjelma valmiiksi aikarajan sisällä. Vaikka henkilökohtaiset harjoitustyöt ovat samanlaisia yhdessä tehtävien harjoitustöiden kanssa, joita opiskelijat ovat jo suorittaneet, ovat Chamillard ja Braun havainneet opiskelijoiden saavan henkilökohtaisista harjoitustöistä huomattavasti heikompia arvosanoja. He myöskin huomasivat opiskelijoiden kaipaavan enemmän apua kurssin asioissa, joita he eivät ole ymmärtäneet henkilökohtaisten harjoitustöiden lähestyessä. Henkilökohtaiset harjoitustyöt näyttävät siis toimivan hyvän ohjelmointitaidon arviointikeinon lisäksi opiskelijoille kannustimena ymmärtää ohjelmointikäsitteitä.

\end{description}

\section{Tilastollista analysointia}

Chamillardin ja Braunin aineisto koostuu yhteensä 1822 opiskelijan pisteytyksistä neljän vuoden kurssien ajalta, syksystä 1997 kevääseen 1999. Loppukokeen tilastoissa on mukana vain 1712 opiskelijan tulokset, sillä loput 110 eivät osallistuneet kokeeseen.

Keskiarvojen ja keskihajontojen perusteella Chamillard ja Braun havaitsivat yhdessä tehtävän harjoitustyön ja ryhmätyön arvosanojen olevan keskimäärin paljon parempia kuin itsenäisesti suoritettavien tehtävien. He arvelevat tämän osin johtuvan itsenäisten tehtävien suorituksen tapahtumisesta valvotussa tilassa aikarajan sisällä. Eroa saattaa muodostua myös itsenäisestä suorituksesta yhteiseen nähden, mutta Chamillard ja Braun eivät tiedä kuinka määrittää tätä vaikutusta.

Chamillard ja Braun haluavat todistaa jokaisen arviointitekniikan mittaavan eri taitoja, tai mittaavaan niitä eri tavoilla. Käytännössä he haluavat nähdä eri arviointitekniikoiden tuottavan erilaiset jakaumat.

Parittainen t-testi on yleinen työkalu tähän tarkoitukseen. Se olettaa kuitenkin vertailtavien jakaumien olevan normaalisti jakautuneita. Kolmogorov-Smirnovin testiä käyttämällä Chamillard ja Braun näkevät että yksikään jakaumista ei ole normaalisti jakautunut, joten t-testiä ei ole mahdollista käyttää. Käyttäen ei-parametristä Merkkitestiä he saavat tilastollisesti merkittävän todisteen siitä, että jokainen arviointitekniikka on toisista eroava.

Chamillard ja Braun pitävät mielenkiintoisena sitä, miten opiskelijoiden suoritukset eri arviointitekniikoiden välillä liittyvät toisiinsa. Tätä tutkiakseen he laskevat jokaiselle arviointitekniikkaparille Pearsonin korrelaatiokertoimen. He havaitsevat jokaisen parin osoittavan tilastollisesti merkittävää lineaarista korrelaatiota.

Vahvoin korrelaatio on testien ja loppukokeen välillä. Chamillard ja Braun eivät pidä tätä yllättävänä, sillä ne ovat hyvin samanalaisia tekniikoita sekä tarkoituksensa että muotonsa puolesta.

Kaksi seuraavaksi vahvinta korrelaatiota ovat yhdessä tehtävät harjoitustyöt ja testit, sekä henkilökohtainen harjoitustyö ja testit. Chamillardin ja Braunin mukaan näiden korrelaatioiden vahvuus kertoo harjoitustöissä hyvin menestyneiden pärjänneen hyvin myös testeissä. Harjoitustöiden vahva korrelaatio loppukokeen kanssa niin ikään tukee teoriaa siitä, että opiskelijat jotka osaavat ohjelmoida, pärjäävät hyvin myös perinteisemmissä kokeissa.

Kaikissa heikoimmissa korrelaatioissa parin toisena osapuolena on ryhmä\-työ. Chamillard ja Braun epäilevät tämän johtuvan muun muassa siitä että ryhmätyössä yksilön panoksella on pienin merkitys arviointiin. Tämän takia he eivät pidä ryhmätyötä oleellisena arvioinnin kannalta, vaan mahdollisuutena antaa opiskelijoiden kokea ryhmässä työskentelyn hyödyt ja haitat.

\section{Yhteenveto}

Ohjelmoinnin alkeiskurssien ohjaajat joutuvat vaikean päätöksen eteen arviointitekniikoita valitessaan. Jotkut tekniikat auttavat enemmän opiskelijoita asian omaksumisessa ja toiset tarjoavat helpommat ja tarkemmat arviointimahdollisuudet.

Asiaan liittyviä avoimia kysymyksiä löytyy edelleen. Esimerkiksi ryhmä\-opiskelun ja itsenäisen opiskelun erojen vertailu identtisillä tehtävillä olisi valaisevaa.

% --- Back matter ---

\bibliographystyle{babplain}
\bibliography{../lahteet}


\end{document}