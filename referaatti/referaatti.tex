\documentclass[finnish]{../tktltiki2}

% --- General packages ---

\usepackage[utf8]{inputenc}
\usepackage{lmodern}
\usepackage{microtype}
\usepackage{amsfonts,amsmath,amssymb,amsthm,booktabs,color,enumitem,graphicx}
\usepackage[pdftex,hidelinks]{hyperref}

% Automatically set the PDF metadata fields
\makeatletter
\AtBeginDocument{\hypersetup{pdftitle = {\@title}, pdfauthor = {\@author}}}
\makeatother

% --- Language-related settings ---

\usepackage[fixlanguage]{babelbib}
\selectbiblanguage{finnish}

\usepackage[nottoc,numbib]{tocbibind}
\settocbibname{Lähteet}

% --- Theorem environment definitions ---

\newtheorem{lau}{Lause}
\newtheorem{lem}[lau]{Lemma}
\newtheorem{kor}[lau]{Korollaari}

\theoremstyle{definition}
\newtheorem{maar}[lau]{Määritelmä}
\newtheorem{ong}{Ongelma}
\newtheorem{alg}[lau]{Algoritmi}
\newtheorem{esim}[lau]{Esimerkki}

\theoremstyle{remark}
\newtheorem*{huom}{Huomautus}


% --- tktltiki2 options ---

\title{Opiskelijoiden ohjelmointitaidon arviointi\\tietojenkäsittelytieteen johdantokurssilla}
\author{Tomi Simsiö}
\date{\today}
\level{Referaatti}

\begin{document}

% --- Front matter ---

\maketitle
\tableofcontents
\newpage


% --- Main matter ---

\section{Johdanto}

Ohjelmointitaidon mittaaminen on vaikeaa. Syynä tähän ei ole keinojen puuttuminen, vaan ohjelmointitaitoa pystytään mittaamaan niin monella tavalla että parhaan tavan valitseminen kuhunkin tilanteeseen on vaikeaa. Chamillard ja Braun esittelevät U.S. Air Force Academyn tietojenkäsittelytieteen johdantokurssilla käytettyjä ohjelmointitaidon arviointitekniikoita artikkelissaan "Evaluating Programming Ability in an Introductory Computer Science Course"~\cite{CB00}. He kertovat kuinka kurssin arvioinnissa yhdistetään useita arviointitekniikoita. 1000 pistettä jaetaan eri tehtävien välillä seuraavasti: yhdessä tehtävä harjoitustyö (180 pistettä), ryhmäprojekti (75 pistettä), henkilökohtainen harjoitustyö (160 pistettä), testejä (300 pistettä), loppukoe (250 pistettä) ja muu aktiivisuus (35 pistettä). Viimeisintä lukuunottamatta kaikki näistä arvioivat opiskelijan ohjelmointitaitoa jollain tapaa.

\section{Yhdessä tehtävät harjoitustyöt}

Ensimmäinen ohjelmointitaidon arviointimetodi on yhdessä tehtävät harjoitustyöt. Opiskelijat tekevät kurssin aikana 6 harjoitustyötä, joista neljälle on varattu 2 tuntia ohjattua aikaa luokassa. Harjoitusten vaikeusaste vaihtelee ensimmäisestä, jossa asetetaan arvoja muuttujiin ja hallitaan yksinkertaisia lukuja ja tulostuksia, viimeiseen harjoitukseen, jossa käytetään taulukoita sekä tiedostolukua ja -kirjoitusta. Harjoitustöiden aikana opiskelijat saavat puhua ja tehdä yhteistyötä toistensa kanssa, mutta jokaisen täytyy palauttaa oma henkilökohtainen työnsä arvostelua varten.

Chamillardin ja Braunin mukaan yhdessä tehtävillä harjoitustöillä tehdään ohjelmoinnin oppimisesta mahdollisimman helppoa. Vaikka paikalla olisi ohjaajiakin, on hyödyllistä sallia opiskelijoiden kysyä neuvoa myös toisiltaan. Joidenkin opiskelijoiden on havaittu oppivan paremmin muiden kanssa. Yhdessä tehtävät harjoitustyöt tarjoavat näille opiskelijoille tehokkaan oppimisympäristön. Muuten hyvän metodin varjopuolena on vaikea arviointi, joka ei anna tarkkaa kuvaa yksilön taitotasosta.


\section{Ryhmätyö}

Ryhmätyö suoritetaan lukukauden loppupuolella 2-4 hengen ryhmissä, Chamillard ja Braun kertovat. Se mahdollistaa opiskelijoiden tutustuttamisen ryhmädynamiikkaan ja tarjoaa kokemuksen projektista, joka on liian iso yhden opiskelijan tehtäväksi. Opiskelijat ovat yleensä toteuttaneet ryhmätyönä jonkin pelin. Kurssin viimeisenä päivänä on järjestetty turnaus jossa opiskelijat pelaavat toisiaan vastaan omia pelejään ja kurssipisteitä jaetaan sijoitusten mukaan. Tämä on innostanut vähemmän lahjakkaitakin opiskelijoita kehittämään strategioita turnauksessa pärjäämiseen.

Ryhmätyö ei sisällä uutta opetussisältöä ja on siksi epäselvää kehittääkö se opiskelijoiden ohjelmointitaitoa, Chamillard ja Braun jatkavat. Anekdoottisen todisteen mukaan ryhmätyö motivoi opiskelijoita ja Chamillard ja Braun uskovat sen tarjoaman kokemuksen olevan hyödyllinen. Yhdessä tehtävän harjoitustyön tavoin ryhmätyökään ei mahdollista yksilöiden ohjelmointitaidon tarkkaa arvioimista.

\section{Henkilökohtaiset harjoitustyöt}

Vaikka yhdessä tehtävässä harjoitustyössä ja ryhmätyössä on ryhmätyöskentelyn tuomat edut, tekee se samalla yksittäisen opiskelijan ohjelmointitaidon arvioinnista vaikeaa, Chamillard ja Braun selostavat. Tämän ongelman ratkaisuna he käyttävät kurssilla kahta henkilökohtaisia harjoitustöitä.

Henkilökohtaisissa harjoitustyössä opiskelijoiden tulee tehdä ja testata annetun ongelman ratkaiseva ohjelma 85 minuutin kuluessa. Avukseen he saavat lehtisen jossa on listattu kurssilla läpikäydyt ohjelmointirakenteet sekä yleisimmät ohjelmointivirheet ratkaisuineen. Kaiken muun materiaalin käyttö on kiellettyä eivätkä ohjaajat anna apua ongelman ratkaisua koskeviin kysymyksiin. Henkilökohtaisilla harjoitustöillä pyritään siis vain arvioimaan opiskelijoiden ohjelmointitaitoa tarkasti. Ensimmäinen henkilökohtainen harjotustyö pidetään kurssin puolivälissä ja toinen noin kolme viikkoa ennen kurssin loppua.

Chamillard ja Braun kertovat näiden harjoitustöiden aiheuttavan stressiä opiskelijoissa. Tavallisen kokeeseen liittyvän stressin lisäksi painetta aiheuttaa tarve saada ohjelma valmiiksi aikarajan sisällä. Vaikka henkilökohtaiset harjoitustyöt ovat samanlaisia yhdessä tehtävien harjoitustöiden kanssa, joita opiskelijat ovat jo suorittaneet, havaitsivat Chamillard ja Braun opiskelijoiden saavan henkilökohtaisista harjoitustöistä huomattavasti heikompia arvosanoja. He myöskin huomasivat opiskelijoiden kaipaavan enemmän apua kurssin asioissa joita he eivät ymmärtäneet henkilökohtaisten harjoitustöiden lähestyessä. Henkilökohtaiset harjoitustyöt näyttävät siis toimivan hyvän ohjelmointitaidon arviointikeinon lisäksi opiskelijoille kannustimena ymmärtää ohjelmointikäsitteitä.

\section{Tilastollista analysointia}

Chamillardin ja Braunin aineisto koostuu yhteensä 1822 opiskelijan pisteytyksistä neljän vuoden kurssien ajalta, syksystä 1997 kevääseen 1999. Loppukokeen tilastoissa on mukana vain 1712 opiskelijan tulokset, sillä loput 110 eivät osallistuneet kokeeseen.

Keskiarvojen ja keskihajontojen perusteella Chamillard ja Braun havaitsivat yhdessä tehtävän harjoitustyön ja ryhmätyön arvosanojen olevan keskimäärin paljon parempia kuin itsenäisesti suoritettavien tehtävien. He arvelevat tämän osin johtuvan itsenäisten tehtävien suorituksen tapahtumisesta valvotussa tilassa aikarajan sisällä. Eroa saattaa muodostua myös itsenäisestä suorituksesta yhteiseen nähden, mutta Chamillard ja Braun eivät tiedä kuinka määrittää tätä vaikutusta.

Chamillard ja Braun haluavat todistaa jokaisen arviointitekniikan mittaavan eri taitoja, tai mittaavaan niitä eri tavoilla. Käytännössä he haluavat nähdä eri arviointitekniikoiden tuottavan erilaiset jakaumat.

Parittainen t-testi on yleinen työkalu tähän tarkoitukseen. Se olettaa kuitenkin vertailtavien jakaumien olevan normaalisti jakautuneita. Kolmogorov-Smirnovin testiä käyttämällä Chamillard ja Braun näkevät että yksikään jakaumista ei ole normaalisti jakautunut, joten t-testiä ei ole mahdollista käyttää. Käyttäen ei-parametristä Merkkitestiä he saavat tilastollisesti merkittävän todisteen siitä että jokainen arviointitekniikka on toisista eroava.

Chamillard ja Braun pitävät myös mielenkiintoisena sitä miten opiskelijoiden suoritukset eri arviointitekniikoiden välillä liittyvät toisiinsa. Tätä tutkiakseen he laskevat jokaiselle arviointitekniikkaparille Pearsonin korrelaatiokertoimen. He havaitsevat jokaisen parin osoittavan tilastollisesti merkittävää lineaarista korrelaatiota.

Vahvoin korrelaatio on testien ja loppukokeen välillä. Chamillard ja Braun eivät pidä tätä yllättävänä, sillä ne ovat hyvin samanalaisia tekniikoita sekä tarkoituksensa että muotonsa puolesta.

Kaksi seuraavaksi vahvinta korrelaatiota ovat yhdessä tehtävät harjoitustyöt ja testit, sekä henkilökohtainen harjoitustyö ja testit. Chamillardin ja Braunin mukaan näiden korrelaatioiden vahvuus kertoo harjoitustöissä hyvin menestyneiden pärjänneen hyvin myös testeissä. Harjoitustöiden vahva korrelaatio loppukokeen kanssa niin ikään tukee teoriaa siitä että opiskelijat jotka osaavat ohjelmoida käytännössä pärjäävät hyvin myös perinteisemmissä kokeissa.

Kaikkissa heikoimmissa korrelaatioissa parin toisena osapuolena on ryhmätyö. Chamillard ja Braun epäilevät tämän johtuvan muun muassa siitä että ryhmätyössä yksilön panoksella on pienin merkitys arviointiin. Tämän takia he eivät pidä ryhmätyötä oleellisena arvioinnin kannalta, mutta mahdollisuutena antaa opiskelijoiden kokea ryhmässä työskentelyn hyödyt ja haitat.

\section{Yhteenveto}

Ohjelmoinnin alkeiskurssien ohjaajat joutuvat vaikean päätöksen eteen arviointitekniikoita valitessaan, Chamillard ja Braun kertovat. Jotkin auttavat enemmän opiskelijoita asian omaksumisessa ja toiset tarjoavat helpommat ja tarkemmat arviointimahdollisuudet.

Asiaan liittyviä avoimia kysymyksiä löytyy edelleen. Esimerkiksi ryhmäopiskelun ja itsenäisen opiskelun erojen vertailu identtisillä tehtävillä olisi valaisevaa, Chamillard ja Braun lopettavat.

% --- Back matter ---

\bibliographystyle{babplain}
\bibliography{../lahteet}


\end{document}